\documentclass[tikz]{standalone} %\usetikzlibrary{calc} 
\usepackage{rubikcube,rubikrotation,rubikpatterns,rubiktwocube} 
\newcommand{\CycleThreeEdgesFlipTwo}{[CycleThreeEdgesFlipTwo],F,R,U,Rp,Up,Fp}%
\newcommand{\cyclethreeedgesfliptwo}{\CycleThreeEdgesFlipTwo}%
\standaloneconfig{border=0bp}
%
%----corner sequences--------------------------
\newcommand{\AllYellow}{[allyellow],R,U,Rp,U,R,Up,Up,Rp}% = SUNE  %cross -->allyellow
\newcommand{\allyellow}{\AllYellow}%
\newcommand{\CycleThreeCorners}{[cyclethreecorners],Lp,U,R,Up,L,U,Rp,Up}%
\newcommand{\cyclethreecorners}{\CycleThreeCorners}%
\newcommand{\cornerrotate}{[cornerrotate],Up,Rp,Dp,R,U,Rp,D,R}
\newcommand{\SwapTwoCorners}{[swaptwocorners],Rp,F,Rp,B2,R,Fp,Rp,B2,R2,Up}
\newcommand{\swaptwocorners}{\SwapTwoCorners}
%
%% brace and bracket macros 
\newcommand{\Rubikbracket}[1]{$\left(\mbox{#1}\right)$}
\newcommand{\Rubikbrace}[1]{$\left\{\mbox{#1}\right\}$}
\newcommand{\cubenumber}[1]{\strut\raisebox{1cm}{#1}}
\definecolor{ptblue}{HTML}{2A5EA4}
\begin{document}
%\begin{figure}[htb]
\centering%
%\RubikFaceUpAll{W}
%\RubikFaceFrontAll{O}
%\RubikFaceRightAll{G}
\RubikCubeSolvedWY
\RubikRotation{random,120}
\ShowCube{5cm}{0.6}{%
\DrawRubikCubeSF%
%%-----------------
%% Right
%\draw[line width=2pt,color=ptblue,<-] (3.5,2) -- (5.3, 2);
%\node (R) at (4.6, 2.5)  [ptblue]{\textbf{\textsf{R}}};
%\node (x) at (5.8, 2)  [ptblue]{\textbf{\textsf{x}}};
%%Left
%\draw[line width=2pt,color=ptblue,<-] (-0.2,2) -- (-1.5, 2);
%\node (L) at (-1.1, 2.5)  [ptblue]{\textbf{\textsf{L}}};
%%Up
%\draw[line width=2pt,color=ptblue,<-] (2, 3.5) -- (2, 5.5);
%\node (U) at (1.4, 4.7)  [ptblue]{\textbf{\textsf{U}}};
%\node (y) at (2, 6.1)  [ptblue]{\textbf{\textsf{y}}};
%%Down
%\draw[line width=2pt,color=ptblue,<-] (2, -0.2) -- (2, -1.5);
%\node (D) at (2.6, -1.1)  [ptblue]{\textbf{\textsf{D}}};
%%Front
%\draw[line width=2pt,color=ptblue,<-] (1.5, 1.5) -- (0, -1);
%\node (F) at (0.7, -0.7)  [ptblue]{\textbf{\textsf{F}}};
%\node (z) at (-0.3, -1.4)  [ptblue]{\textbf{\textsf{z}}};
%%Back
%\draw[line width=2pt,color=ptblue,<-] (3.2, 4.2) -- (4, 5.5);
%\node (B) at (4.4, 5)  [ptblue]{\textbf{\textsf{B}}};
%%
}
%\end{figure}
\end{document}