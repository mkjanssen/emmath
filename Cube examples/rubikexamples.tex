%%
%% rubikexamples.tex 
%% February 25, 2018
%% Requires ALL the rubik bundle packages and also the tikz package
%% Part of the Rubik bundle v5.0 (www.ctan.org/pkg/rubik)
%% authors: RWD Nickalls & A Syropoulos
%%--------NOTE-------------------------------------------
%% USAGE:   (pdf)latex  --shell-escape   rubikexamples.tex
%% run three times to get all labels and fig nos correct
%%-------------------------------------------------------
\documentclass[a4paper]{article}

%------rubik----------
\usepackage{tikz}   %% load tikz BEFORE rubikcube
\usepackage{rubikcube,rubikrotation,rubikpatterns,rubiktwocube} 
%----------------------
\usepackage{url}
\usepackage{ifpdf}
\usepackage{lmodern}
%---------------------
\ifpdf
  \usepackage[verbose]{microtype}
  \usepackage{cmap}
  \usepackage[pdfencoding=auto]{hyperref}

  \hypersetup{%
     pdftitle={rubikexamples.pdf},
     pdfsubject={Rubik cube, Rubik bundle},
     pdfkeywords={Rubik cube LaTeX}
     }
\fi
%%-------------------
\pagestyle{myheadings}
\markright{\texttt{rubikexamples.pdf} \ \ 
(Rubik bundle v5.0, 2018) \ \ \texttt{www.ctan.org/pkg/rubik}}

%% increase text height and width  
\addtolength{\oddsidemargin}{-1.5cm}
\addtolength{\textwidth}{3cm}
\addtolength{\textheight}{1cm}
%----------------------
%==========Rubik stuff================
%----edge sequences for 3x3x3 cubes--------------------------
\newcommand{\CycleThreeEdgesFlipTwo}{[CycleThreeEdgesFlipTwo],F,R,U,Rp,Up,Fp}%
\newcommand{\cyclethreeedgesfliptwo}{\CycleThreeEdgesFlipTwo}%
\newcommand{\CornerRotation}{[CornerRotation],Up,Rp,Dp,R,U,Rp,D,R}
\newcommand{\cornerrotation}{\CornerRotation}
%
%----corner sequences--------------------------
\newcommand{\AllYellow}{[allyellow],R,U,Rp,U,R,Up,Up,Rp}% = SUNE  %cross -->allyellow
\newcommand{\allyellow}{\AllYellow}%
\newcommand{\CycleThreeCorners}{[cyclethreecorners],Lp,U,R,Up,L,U,Rp,Up}%
\newcommand{\cyclethreecorners}{\CycleThreeCorners}%
\newcommand{\SwapTwoCorners}{[swaptwocorners],Rp,F,Rp,B2,R,Fp,Rp,B2,R2,Up}
\newcommand{\swaptwocorners}{\SwapTwoCorners}
%
%% brace and bracket macros 
\newcommand{\Rubikbracket}[1]{$\left(\mbox{#1}\right)$}
\newcommand{\Rubikbrace}[1]{$\left\{\mbox{#1}\right\}$}
\newcommand{\cubenumber}[1]{\strut\raisebox{1cm}{#1}}
%-----------------
%
\begin{document}
%
\ifpdf\pdfbookmark[1]{Title}{Title}\fi
\title{Rubik examples\\ 
\smallskip\normalsize\texttt{www.ctan.org/tex-archives/macros/latex/contrib/rubik/rubikexamples.pdf}
\,\footnote{This file is part of the Rubik bundle v5.0. To generate this file, 
use the following command: \newline \texttt{\$ pdflatex --shell-escape  rubikexamples.tex}}}
\author{RWD Nickalls\,\footnote{email: \textsf{dick@nickalls.org}}%
  \ \ \& A Syropoulos\,\footnote{email: \textsf{asyropoulos@yahoo.com}}}
\date{25 February, 2018 (Rubik bundle v5.0)}
\maketitle
\tableofcontents

\pagebreak
%%----------------------
\section{Preliminaries}
These  examples  were generated using  the \TeX\ 
Rubik bundle\,\footnote{\url{http://www.ctan.org/pkg/rubik}} v5.0. 
They assume some familiarity with the four  packages \textsc{rubikcube}, 
\textsc{rubikrotation}, \textsc{rubikpatterns} and \textsc{rubiktwocube}. 
For~documentation see the following  files:

\begin{quote}
\begin{verbatim}
rubikcube.pdf           3x3x3 documentation
rubikrotation.pdf       commands for implementing rotation sequences
rubikrotationPL.pdf     documentation of rubikrotation.pl (Perl program)
rubikpatterns.pdf       small database of well-known 3x3x3 configurations
rubikpatternsLIST.pdf   shows all configurations in the rubikpatterns database
rubiktwocube.pdf        2x2x2 documentation
\end{verbatim}
\end{quote}

This file requires the following packages: \texttt{tikz, rubikcube, rubikrotation,
rubikpatterns, rubiktwocube}; note that the \texttt{tikz} package must be loaded 
\textit{before} the \texttt{rubikcube} package.

This file needs to be run using  the \verb!--shell-escape! command-line option; 
for example:
\begin{quote}
\begin{verbatim}
pdflatex  --shell-escape  rubikexample.tex
\end{verbatim}
\end{quote}
This is because nearly all the examples  make use of the  \verb!\RubikRotation! 
command, which  calls the Perl script  \texttt{rubikrotation.pl}.  
If you do forget to use the \verb!--shell-escape!  command-line switch, 
the file will still run, but all the cubes will remain in the initial 
unprocessed configuration\,\footnote{As a quick check, run this file 
(\texttt{rubikexamples.tex}) in a separate directory and look at Figure~1; 
if this appears as a `solved' cube then your system has failed to call the Perl 
script \texttt{rubikrotation.pl} correctly for some reason. The most likely 
causes of this are (a)~failure to invoke the \texttt{--shell-escape}  command-line 
option correctly, or (b)~failure to use a configuration file correctly 
(ie your system can't find the Perl program)---search the \texttt{.log} file 
to see whether the file \texttt{rubikrotation.pl} has been used.}. 

\subsection{3x3x3 vs 2x2x2 command names}
The introduction of the  \textsc{rubiktwocube} package  (Rubik bundle v5)
has necessitated a few command name changes in order to avoid confusion 
(see the \textsc{rubikcube} package documentation \S\,3.1 for details). 
Consequently, command names containing the word `\texttt{Rubik}'  denote 
a 3x3x3 cube action; similarly those containing the word `\texttt{Two}'  
denote a 2x2x2 cube action. 

\subsection{Environments}
When using the Rubik bundle one sometimes needs to be mindful of the various 
\LaTeX\ environments in which Rubik commands are placed (e.g.,~the figure, 
minipage and TikZ picture environments), since these environments restrict 
the actions of commands they contain to the particular environment. 
The \verb!\ShowCube! command is also relevant here, since it is a 
minipage-wrapper for the TikZ picture environment. Only Rubik  \verb!\Draw..! 
commands and TikZ commands (e.g.,~\verb!\draw! and \verb!\node!) 
actually need to be inside a TikZ picture environment, and hence 
inside a \verb!\ShowCube! command. 

This issue arises because the Rubik bundle allows you to create figures 
showing different stages  during a sequence of rotations. Consequently 
the effects of  commands executed inside an environment (especially 
commands which determine the colour-state or rotations), may not  be 
apparent to subsequent commands outside that particular environment. 
See Example~3  for an illustration of how to handle environments.

\pagebreak

%%------------------
\section{RubikCube examples (3x3x3)}

\subsection{Sixspot}
\typeout{---Example (sixspot)}

In Figure~\ref{fig:sixspot} we show  the so-called ``sixspot'' configuration,
 generated from a solved cube using the rotation sequence 
\rr{U}, \rr{Dp}, \rr{R}, \rr{Lp}, \rr{F}, \rr{Bp}, \rr{U}, \rr{Dp}.

%%--------FIGURE 1 -----------
\begin{figure}[hbt]
\centering
\RubikCubeSolvedWY
\RubikRotation{\sixspot}
\ShowCube{7cm}{0.7}{\DrawRubikCubeSF}
\caption{\label{fig:sixspot}The `sixspot' configuration.}
\end{figure}
%----------------

{\noindent}Creating a macro to hold  a rotation sequence greatly 
facilitates their use, as follows:
\begin{quote}
\begin{verbatim}
\newcommand{\sixspot}{[sixspot],U,Dp,R,Lp,F,Bp,U,Dp}
\end{verbatim}
\end{quote}
We can now process this sequence  using its macro name as an argument 
for the \verb!\RubikRotation! command. The code used for the above 
Figure  uses the \verb!\ShowCube{}{}{}! command for which \verb!#1! 
is the minipage width, \verb!#2! is the tikzpicture scale factor (0--1), 
and \verb!#3!  can include  \textsc{rubikcube} package  \verb!\Draw..! 
commands,  and any commands which are valid for use in a TikZ 
\texttt{tikzpicture} environment. The code for the above figure is 
as follows:

\begin{quote}
\begin{verbatim}
\begin{figure}[hbt]
   \centering
   \RubikCubeSolvedWY
   \RubikRotation{\sixspot}
   \ShowCube{7cm}{0.7}{\DrawRubikCubeSF}
\caption{...}
\end{figure}
\end{verbatim}
\end{quote}

Note that the sixspot sequence is a so-called `order~3' sequence, which 
means that running the `sixspot' sequence 3 times returns the cube 
to its original `solved' state. The~command for processing it three 
times is  \verb!\RubikRotation[3]{\sixspot}!.

Note that the semi-flat form of the cube here is generated by the
 \verb!\DrawRubikCubeSF! command, where the terminal SF denotes 
 the Semi-Flat form.  To draw it completely flat use a terminal F, 
 e.g.,~\verb!\DrawRubikCubeF!  (see Figure~\ref{fig:sixspotE}). 

\subsubsection{Log-file extract}
Users may find it instructive to inspect the the log-file and 
follow the dynamic interaction between \LaTeX\  and the Perl script. 
This is easy to follow, since output by \texttt{rubikrotation.sty} 
is prefixed by 3 dashes (\verb!---!), while output by the Perl script 
is prefixed by 3 dots (\verb!...!). Search for the keyword `Example', 
as this is written to the log-file at the start of each example. 

The following is part of the log-file extract associated with Example~1. 

\begin{quote}
\begin{verbatim} 
---Example (sixspot)

---TeX process---------------------------
---script = rubikrotation.sty v5.0 (2018/02/25)
---NEW rotation command
---command = RubikRotation[1]{[SixSpot],U,Dp,R,Lp,F,Bp,U,Dp,<(8q*, 8f*)>}
---writing current cube state to file rubikstate.dat
\openout7 = `rubikstate.dat'.

\ourRRcounter=\count134
---CALLing Perl script (rubikrotation.pl)
runsystem(perl $HOME/RubikTeX/PERLstuff/dtx-rubikrotation/rubikrotation.pl
 -i rubikstate.dat -o rubikstateNEW.dat)...executed.

---inputting NEW datafile (data written by Perl script)
(./rubikstateNEW.dat

...PERL process..................................
...script = rubikrotation.pl v5.0 (25 February 2018)
...reading the current cube state (from File: rubikstate.dat)
...
...command = cubesize,three
...cube = THREEcube
...
......up,W,W,W,W,W,W,W,W,W
....down,Y,Y,Y,Y,Y,Y,Y,Y,Y
....left,B,B,B,B,B,B,B,B,B
...right,G,G,G,G,G,G,G,G,G
...front,O,O,O,O,O,O,O,O,O
....back,R,R,R,R,R,R,R,R,R
...
...rotation keyword
...checking state of cube
...cubiesum = 54 (Red=9, Or=9, Ye=9, Gr=9, Bl=9, Wh=9, X=0)
...
...command = rotation,[SixSpot],U,Dp,R,Lp,F,Bp,U,Dp,<(8q*, 8f*)>
...dataline = rotation,[SixSpot],U,Dp,R,Lp,F,Bp,U,Dp,<(8q*; 8f*)>
...[SixSpot] is a label OK
...rotation U, OK
...rotation Dp, OK 
...rotation R, OK
...rotation Lp, OK
...rotation F, OK
...rotation Bp, OK
...rotation U, OK
...rotation Dp, OK 
...writing new cube state to file rubikstateNEW.dat
...SequenceName = SixSpot
...SequenceInfo = (8q*; 8f*)
...SequenceShort = [SixSpot],U,Dp,R,Lp,F,Bp,U,Dp
...SequenceLong = U,Dp,R,Lp,F,Bp,U,Dp
)
\end{verbatim}
\end{quote}

\pagebreak

\subsection{ShowErrors}
\typeout{---Example (ShowRubikErrors)}

 In this example we demonstrate  the use of the  \verb!\ShowErrors! 
 command\,\footnote{  The \texttt{ {\textbackslash}ShowErrors} command 
 replaces the earlier \texttt{{\textbackslash}ShowRubikErrors} command 
 which is now deprecated. However, the earlier command is still
 supported (for now) for backward compatibility.}, 
 which  places a copy of  the Perl output file \verb!rubikstateERRORS.dat! 
 underneath the graphic so you can see a list of all the  errors, if any. 
 Note that this example is similar to the previous one except that we have  
 introduced several  errors---e.g.,~bad minipage width,  typos, as well as 
 some animals---into  the rotation sequence). 
 It is important to note that  the  \verb!\ShowErrors! command must be 
 placed \textit{after}  the TikZ picture environment (i.e.,~in this case after 
the \verb!\ShowCube! command), or even at the end of the document. 
Note that full details of all errors are also included in the .log file (search for \texttt{*ERR}). 

%%--------FIGURE 2 -----------
%
\RubikCubeSolvedWY
\RubikRotation{[sixspot],U,Dp,R,Lp,F,Bp,U,Dpppp,cat,dog}
\begin{figure}[hbt]
\centering
\ShowCubeF{4.5cm}{0.5}{\DrawRubikCubeF}
\ShowErrors
\parbox{0.8\textwidth}{%
\caption{\label{fig:sixspotE}The same `sixspot' sequence of rotations as shown 
in Example~1,  but now with some   errors (wrong minipage width, typos and some 
animals\,!) in the rotation sequence (it \textit{should} be just 
\texttt{U,Dp,R,Lp,F,Bp,U,Dp}).}}
\end{figure}
%
{\noindent}In this example we have used the F version of the \verb!\ShowCube! 
command (\verb!\ShowCubeF!)  which places an fbox around the image so you can see the 
extent of any white space etc. This reveals that the set minipage-width ($4.5$cm) in 
the \verb!\ShowCubeF! command---see code below---is too narrow: it should be 
5cm ($10 \times 0.5$) to just include the whole image (i.e.,~$10 \times$ the TikZ 
scale-factor in this case). 
Once fixed, we can remove the F from the \verb!\ShowCubeF! command.
Note also  that only `\verb!\Draw...!' commands really need to be inside the TikZ 
picture environment (i.e.,~inside the \verb!\ShowCube! command). The above figure 
was generated by the following code.
%-----------------
\begin{quote}
\begin{verbatim}
\RubikCubeSolvedWY
\RubikRotation{[sixspot],U,Dp,R,Lp,F,Bp,U,Dpppp,cat,dog}
\begin{figure}[hbt]
  \centering
  \ShowCubeF{4.5cm}{0.5}{\DrawRubikCubeF}
  \ShowErrors
\caption{....}
\end{figure}
\end{verbatim}
\end{quote}
Even if the \verb!\ShowErrors! command is not used, it is always a good idea 
to check the file \verb!rubikstateERRORS.dat! after a \LaTeX\ run, since this 
file will also reveal any  errors.


\pagebreak
%----------------------

\subsection{Environments}
\typeout{---Example (Environments)}


In this example we highlight the fact that Rubik commands used inside a \LaTeX\ 
environment remain local to that environment, and how this can sometimes  be 
problematic.  Commands whose reach is meant to be  more global need to be 
executed outside such environments, where they can implement  global colour 
settings, which will then be accessible to subsequent \verb!\Draw..! commands.

Since we are drawing images, this is primarily an issue 
with the  \verb!minipage!, \verb!figure!, and TikZ picture environments.
Consequently, it is generally best when drawing a sequence of  cubes  to reserve 
the TikZ picture environment  only  for  \verb!\Draw..! commands and TikZ 
commands. Importantly, this  also applies to our main display tool the \verb!\ShowCube! 
command, since this is a minipage-wrapper for the TikZ  picture environment 
(see the \textsc{rubikcube} package documentation).

In this example the first cube (9a) uses a \verb!\RubikCubeSolvedWY! command 
\textit{inside} a \verb!\ShowCube! environment. However, if we now perform the 
rotation  \textRubik{R}\ (using the command \verb!\RubikRotation{R}!) this results 
in  a quite unexpected effect on cube (9b) (and obviously not correct). 
This is because  the action of the initial \verb!\RubikCubeSolved! command 
 (setting a new colour-state) is not  visible outside its \verb!\ShowCube! 
 environment, and hence the subsequent \verb!\RubikRotation{R}! command is unaware 
 of this earlier attempt to update  the global colour-state information. 
 It turns out that the internal colour-state was actually last updated following 
 the action of the \textbackslash\texttt{RubikRotation\{[sixspot],....\}} 
 command used in Example~2, being the last colour-state command executed 
 \textit{outside} an environment (a \texttt{figure}  environment in that 
 example). Consequently, the strange cube 9b is not what we expected.

\medskip
\noindent%
\cubenumber{9a}%
\ShowCube{2cm}{0.5}{%
    \RubikCubeSolvedWY%
    \DrawRubikCubeRU}%
\quad\Rubik{R}%    
\RubikRotation{R}%
\cubenumber{9b}%
\ShowCube{2cm}{0.5}{\DrawRubikCubeRU}%
%---------------
\hspace{5mm}
\begin{minipage}{0.5\textwidth}
\begin{verbatim}
\usepackage{tikz,rubikcube,rubikrotation}
\newcommand{\cubenumber}[1]{\strut\raisebox{1cm}{#1}}
...
\cubenumber{9a}%
\ShowCube{2cm}{0.5}{%
    \RubikCubeSolvedWY%
    \DrawRubikCubeRU%
     }%
\quad\Rubik{R}%
\RubikRotation{R}%
\cubenumber{9b}%
\ShowCube{2cm}{0.5}{\DrawRubikCubeRU}%
\end{verbatim}
\end{minipage}

\bigskip

If we now bring the \verb!\RubikCubeSolvedWY! command out of the \verb!\ShowCube! 
environment and place it \textit{before} the  \verb!\ShowCube! command (see cubes 9c, 9d), 
then its colour `state' information  becomes globally accessible (i.e.,~colour-state 
gets updated) and therefore  gets used by the subsequent  \verb!\RubikRotation{R}! 
command, and  hence the cube (9d)  is now rendered correctly.

\bigskip

\noindent%
\cubenumber{9c}%
\RubikCubeSolvedWY%
\ShowCube{2cm}{0.5}{\DrawRubikCubeRU}%
\quad\Rubik{R}%
\RubikRotation{R}%
\cubenumber{9d}%
\ShowCube{2cm}{0.5}{\DrawRubikCubeRU}%
%---------------
\hspace{5mm}
\begin{minipage}{0.5\textwidth}
\begin{verbatim}
\cubenumber{9c}%
\RubikCubeSolvedWY%
\ShowCube{2cm}{0.5}{\DrawRubikCubeRU}%
\quad\Rubik{R}%
\RubikRotation{R}%
\cubenumber{9d}%
\ShowCube{2cm}{0.5}{\DrawRubikCubeRU}%
\end{minipage}
\end{verbatim}
\end{minipage}

\bigskip

Note that the horizontal space occupied by \texttt{9d} is almost exactly 
the same as \verb!\quad!, and so we don't need to include any additional 
space before the \verb!\cubenumber{9d}! command.



\pagebreak
%----------------------
\subsection{Coordinates}
\typeout{---Example (coordinates)}


For all cubes the origin of coordinates is defined as the bottom left corner of 
the \textsc{front} face. Consequently, it is easy to determine the coordinates 
of points and hence draw lines, circles, and place lettering or other objects 
using the standard TikZ \verb!\draw..! and \verb!\node..! commands. Note that for 
convenience  point $P$ (bottom back left corner) is designed to be $(-1,-1)$ 
on the 2D view. (The~following diagram  is Fig~1 from the \textsc{rubikcube} 
package documentation).

\begin{figure}[hbt]
\centering
\RubikFaceFrontAll{X}   % X = default = grey
\ShowCube{5.6cm}{0.7}{%
  \DrawRubikFaceFront

\draw[line join=round,line cap=round,ultra thick] (0,0) -- (0,4); % Y
\draw[line join=round,line cap=round,ultra thick] (0,0) -- (4,0); % X

%draw labels
\node (Ylabel) at (-0.35,  3.8)  {$Y$};
\node (Xlabel) at ( 3.8,  -0.4)  {$X$};

%%----------draw the Rubik faces--------
%% outline Front face
\draw[line join=round,line cap=round,ultra thick]%
     (0,3) -- (-1,2) -- (-1,-1) -- (2,-1) -- (3,0);
%%
%% outline  Down face
\draw[line join=round,line cap=round,ultra thick]%
     (-1,-1) -- (0, 0);
%%
%%
%% draw labels
\node (Y0) at (-0.4, 0)  [blue]{$0$};
\node (Y1) at (-0.4, 1)  [blue]{$1$};
\node (Y2) at (-0.4, 2)  [blue]{$2$};
\node (Y3) at (-0.4, 3)  [blue]{$3$};
%%
\node (X0) at (0, -0.5)  [blue]{$0$};
\node (X1) at (1, -0.5)  [blue]{$1$};
\node (X2) at (2, -0.5)  [blue]{$2$};
\node (X3) at (3, -0.5)  [blue]{$3$};
%%
\node (P) at (-2.4, -1)  [blue]{$P\,(-1,-1)$};
}
\end{figure}

{\noindent}The code for the figure is given below.

We  draw everything in the  \verb!\ShowCube! environment; the \textsc{front} face
in grey (colour code = \texttt{X}) using the Rubik command 
\verb!\DrawRubikFlatFront!, and then draw all the lines and text using standard 
TikZ commands. The correct  minipage-width argument (5.6cm) for the 
\verb!\ShowCube! command is determined by trial-and-error, using the `fbox' 
form of the command (\verb!\ShowCubeF!), and then the `F' is removed 
($\rightarrow$ \verb!\ShowCube!). In order to avoid confusion, all Rubik commands 
start with a capital letter (e.g.,~\verb!\Draw..!), while all TikZ commands 
start with a lower-case letter (e.g.,~\verb!\draw..!).

\begin{quote}
\begin{verbatim}
\begin{figure}[hbt]
\centering
\RubikFaceFrontAll{X}% X = default non-colour (grey)
\ShowCube{5.6cm}{0.7}{%
   \DrawRubikFaceFront
   \draw[line join=round,line cap=round,ultra thick] (0,0) -- (0,4);% Yaxis
   \draw[line join=round,line cap=round,ultra thick] (0,0) -- (4,0);% Xaxis
   \node (Ylabel) at (-0.35,  3.8)  {$Y$};
   \node (Xlabel) at ( 3.8,  -0.4)  {$X$};
   %% outline Left and Down faces
   \draw[line join=round,line cap=round,ultra thick]%
        (0,3) -- (-1,2) -- (-1,-1) -- (2,-1) -- (3,0);
   \draw[line join=round,line cap=round,ultra thick]%
        (-1,-1) -- (0, 0);
   \node (Y0) at (-0.4, 0)  [blue]{$0$};
   \node (Y1) at (-0.4, 1)  [blue]{$1$};
   \node (Y2) at (-0.4, 2)  [blue]{$2$};
   \node (Y3) at (-0.4, 3)  [blue]{$3$};
   \node (X0) at (0, -0.5)  [blue]{$0$};
   \node (X1) at (1, -0.5)  [blue]{$1$};
   \node (X2) at (2, -0.5)  [blue]{$2$};
   \node (X3) at (3, -0.5)  [blue]{$3$};
   \node (P) at (-2.4, -1)  [blue]{$P\,(-1,-1)$};
   }
\end{figure}
\end{verbatim} 
\end{quote}


\pagebreak

%----------------------
\subsection{Face notation}
\typeout{---Example (face notation)}

The following diagram  is Fig~2 from the \textsc{rubikcube} package documentation.

\begin{figure}[htb]
\centering%
\RubikFaceUpAll{Y}
\RubikFaceFrontAll{G}
\RubikFaceRightAll{O}
\ShowCubeF{5cm}{0.6}{%
\DrawRubikCubeRU%
%%-----------------

%% Right
\draw[line width=2pt,color=blue,->] (3.5,2) -- (5.3, 2);
\node (R) at (4.6, 2.5)  [blue]{\textbf{\textsf{R}}};
\node (x) at (5.8, 2)  [blue]{\textbf{\textsf{x}}};

%%Left
\draw[line width=2pt,color=blue] (-0.2,2) -- (-1.5, 2);
\node (L) at (-1.1, 2.5)  [blue]{\textbf{\textsf{L}}};

%%Up
\draw[line width=2pt,color=blue,->] (2, 3.5) -- (2, 5.5);
\node (U) at (1.4, 4.7)  [blue]{\textbf{\textsf{U}}};
\node (y) at (2, 6.1)  [blue]{\textbf{\textsf{y}}};

%%Down
\draw[line width=2pt,color=blue] (2, -0.2) -- (2, -1.5);
\node (D) at (2.6, -1.1)  [blue]{\textbf{\textsf{D}}};


%%Front
\draw[line width=2pt,color=blue,->] (1.5, 1.5) -- (0, -1);
\node (F) at (0.7, -0.7)  [blue]{\textbf{\textsf{F}}};
\node (z) at (-0.3, -1.4)  [blue]{\textbf{\textsf{z}}};

%%Back
\draw[line width=2pt,color=blue] (3.2, 4.2) -- (4, 5.5);
\node (B) at (4.4, 5)  [blue]{\textbf{\textsf{B}}};
%%
}
\end{figure}

{\noindent}The code for the figure is as follows 
(the origin $(0,0)$ is at the bottom left corner of the \textsc{front} (green) face).

\begin{quote}
\begin{verbatim}
\begin{figure}[htb]
\centering%
\RubikFaceUpAll{Y}
\RubikFaceFrontAll{G}
\RubikFaceRightAll{O}
\ShowCubeF{5cm}{0.6}{%
  \DrawRubikCubeRU%
  %% Right
  \draw[line width=2pt,color=blue,->] (3.5,2) -- (5.3, 2);
  \node (R) at (4.6, 2.5)  [blue]{\textbf{\textsf{R}}};
  \node (x) at (5.8, 2)  [blue]{\textbf{\textsf{x}}};
  %%Left
  \draw[line width=2pt,color=blue] (-0.2,2) -- (-1.5, 2);
  \node (L) at (-1.1, 2.5)  [blue]{\textbf{\textsf{L}}};
  %%Up
  \draw[line width=2pt,color=blue,->] (2, 3.5) -- (2, 5.5);
  \node (U) at (1.4, 4.7)  [blue]{\textbf{\textsf{U}}};
  \node (y) at (2, 6.1)  [blue]{\textbf{\textsf{y}}};
  %%Down
  \draw[line width=2pt,color=blue] (2, -0.2) -- (2, -1.5);
  \node (D) at (2.6, -1.1)  [blue]{\textbf{\textsf{D}}};
  %%Front
  \draw[line width=2pt,color=blue,->] (1.5, 1.5) -- (0, -1);
  \node (F) at (0.7, -0.7)  [blue]{\textbf{\textsf{F}}};
  \node (z) at (-0.3, -1.4)  [blue]{\textbf{\textsf{z}}};
  %%Back
  \draw[line width=2pt,color=blue] (3.2, 4.2) -- (4, 5.5);
  \node (B) at (4.4, 5)  [blue]{\textbf{\textsf{B}}};
  }
\end{figure} 
\end{verbatim}
\end{quote}


\pagebreak

%----------------------
\subsection{Grey cube}
\label{sec:greycube}
\typeout{---Example (grey cube)}

When explaining elementary layer-1 moves, it can be useful to use the `grey cube' 
(\verb!\RubikCubeGreyWY!) ---White opposite Yellow--- as this  sets up only 
the central cubie on each face. 
We have shown the first cube here in Semi-Flat (SF) mode simply to show  how the grey 
cube is configured  (note that  a WB version (\verb!\RubikCubeGreyWB!) 
and an `all-grey' version (\verb!\RubikCubeAllGrey!)  are also available. 
Note that all these grey cube commands will also  accept 
the word `gray' (to be consistent with TikZ).

In this  example, we show how to position a single  `flipped'  white/orange edge 
cubie in the top  layer. 
  
\begin{figure}[hbt]

\centering%
\RubikCubeGreyWY%

\RubikFaceUp{X}{X}{X}%
            {X}{W}{O}%
            {X}{X}{X}%
               
\RubikFaceRight{X}{W}{X}
               {X}{G}{X}
               {X}{X}{X}
%
\cubenumber{1}%
\ShowCube{5cm}{0.5}{\DrawRubikCubeSF}%
%------------------
\quad\Rubik{Rp}\RubikRotation{Rp}
\cubenumber{2}%
\ShowCube{2cm}{0.5}{\DrawRubikCube}%
%-----------------------
\quad\Rubik{Fp}\RubikRotation{Fp}
\cubenumber{3}%
\ShowCube{2cm}{0.5}{\DrawRubikCube}%
%-----------------------
\end{figure}


{\noindent}The code for the figure is given below. 
After setting up the first cube, we then just use the \verb!\RubikRotation! 
command to generate the  remaining cubes.
The colours are coded as follows:  R~(red), O~(orange), Y~(yellow), 
G~(green), B~(blue),  W~(white),  and   X~(grey).

\begin{quote}
\begin{verbatim}
\usepackage{tikz,rubikcube,rubikrotation}
\newcommand{\cubenumber}[1]{\strut\raisebox{1cm}{#1}}
...
\begin{figure}[hbt]
\centering
% set up the first cube
\RubikCubeGreyWY%
\RubikFaceUp{X}{X}{X}%
            {X}{W}{O}%
            {X}{X}{X}%

\RubikFaceRight{X}{W}{X}
               {X}{G}{X}
               {X}{X}{X}
\cubenumber{1}%
\ShowCube{5cm}{0.5}{\DrawRubikCubeSF}%
%
\quad\Rubik{Rp}\RubikRotation{Rp}
\cubenumber{2}%
\ShowCube{2cm}{0.5}{\DrawRubikCube}%
%
\quad\Rubik{Fp}\RubikRotation{Fp}
\cubenumber{3}%
\ShowCube{2cm}{0.5}{\DrawRubikCube}%
\end{figure} 
\end{verbatim}
\end{quote}


\pagebreak

%-------------------------------
\subsection{Scramble a cube}
\label{sec:scrambleacube}
\typeout{---Example (scramble a cube)}

In this example we use the \verb!\RubikRotation! command to scramble a `solved' 
Rubik cube via  a sequence of 120 random rotations, using the following  command 
(the details of the process can be seen in the \texttt{.log} file):
\begin{quote}
\begin{verbatim}
\RubikRotation{random,120}
\end{verbatim}
\end{quote}
On this occasion we draw the cube Flat (F) using the command \verb!\DrawRubikCubeF!.
In this example, we also make use of the \verb!\SaveRubikState{}! command to  save 
the  configuration (state) displayed here in Figure~3 to a file 
(\texttt{rubikexampfig3.tex}) using \verb!\SaveRubikState{rubikexampfig3.tex}!,
so we can  display the  same cube configuration later  but in  a different format
(we show it again in the following example (Example~\ref{sec:saverubikstate}).
Note that since we are using a random sequence, it follows that each 
time this file is run not only will a visually different  cube  be generated,  
but the same state  will be shown  both 
here and in Example~\ref{sec:saverubikstate}.
%%--------FIGURE 3 -----------
\begin{figure}[hbt]
  \centering
  \RubikCubeSolvedWY
  \RubikRotation{random,120}
  \SaveRubikState{rubikexampfig3.tex}  
  \ShowCubeF{6cm}{0.5}{\DrawRubikCubeF}
\parbox{0.7\textwidth}{%
\caption{\label{fig:randomone}This shows a cube  generated by 120 random rotations}}
\end{figure}
%----------------
\begin{quote}
\begin{verbatim}
\usepackage{tikz,rubikcube,rubikrotation}
...
\begin{figure}[hbt]
  \centering\RubikCubeSolvedWY
  \RubikRotation{random,120}
  \SaveRubikState{rubikexampfig3.tex}  
  \ShowCubeF{6cm}{0.5}{\DrawRubikCubeF}
\caption{....}
\end{figure}
\end{verbatim}
\end{quote}

{\noindent}Q: How do we determine the minipage-width ($7.2$cm) in the 
\verb!\ShowCube! command?

\smallskip

{\noindent}A: The object is 12~cubie squares  wide. Since the TikZ scale-factor 
argument of the  \verb!\ShowCube! command (cms/unit length) in this 
case is set to~$0.5$, then the true width of  the image will 
be $12 \times 0.5 = 6$\,cm. Note that here we have used the \verb!\ShowCubeF!
command and so we can see that this is correct. Changing the scale-factor will 
change the image size and hence a new width argument will be required to just fit 
the image.  

Note that in this particular case (where there is only a single image in the 
`figure' environment), since the \verb!\ShowCube! command places the image 
(in a TikZ picture environment) centrally  inside a minipage, the image will 
in fact be centrally placed in  the \verb!\textwidth! provided the image 
is \textit{smaller} than the fbox---i.e.,~if we used instead a minipage-width of, 
say, 12\,cm  the image would still  appear centred in the \verb!\textwidth! in 
this case. However, when there are several images in the `figure', then the spacing 
may appear strange unless each image closely fits its own minipage-width etc.
It is often useful, therefore, to check the size of the fbox (using the 
\verb!\ShowCubeF! command) as we have done here.
   
 
\pagebreak

%----------------------
\subsection{SaveRubikState}
\label{sec:saverubikstate}
\typeout{---Example (SaveRubikState)}

In this example we display a cube having the same colour state as that shown in 
the previous example (Example~\ref{sec:scrambleacube}) ---see Figure~\ref{fig:randomone}. 
The cube's colour state was saved from the previous example using the  command 
\verb!\SaveRubikState{rubikexampfig3.tex}!, and then input here  
using the command \verb!%% filename: rubikexampfig3.tex%
\RubikFaceUp{R}{O}{G}{R}{R}{O}{W}{W}{G}%
\RubikFaceDown{W}{B}{B}{R}{O}{B}{B}{R}{B}%
\RubikFaceLeft{G}{W}{O}{Y}{B}{O}{Y}{B}{B}%
\RubikFaceRight{Y}{Y}{Y}{Y}{G}{R}{R}{O}{W}%
\RubikFaceFront{G}{G}{O}{G}{W}{B}{R}{W}{Y}%
\RubikFaceBack{R}{W}{W}{G}{Y}{G}{O}{Y}{O}%
!. These commands therefore allow  the 
state of a previous cube to be saved to a file, and then  displayed again 
later in  a different format.

%%-----------FIGURE 4 -----------
\begin{figure}[hbt]
  \centering
  %% filename: rubikexampfig3.tex%
\RubikFaceUp{R}{O}{G}{R}{R}{O}{W}{W}{G}%
\RubikFaceDown{W}{B}{B}{R}{O}{B}{B}{R}{B}%
\RubikFaceLeft{G}{W}{O}{Y}{B}{O}{Y}{B}{B}%
\RubikFaceRight{Y}{Y}{Y}{Y}{G}{R}{R}{O}{W}%
\RubikFaceFront{G}{G}{O}{G}{W}{B}{R}{W}{Y}%
\RubikFaceBack{R}{W}{W}{G}{Y}{G}{O}{Y}{O}%

  \ShowCube{4cm}{0.8}{\DrawRubikCubeLD}
\parbox{0.7\textwidth}{%
  \caption{\label{fig:randomtwo} This shows a Rubik cube in exactly the  same 
  colour state as  the one shown in Figure~\ref{fig:randomone}, but here displayed 
  as a cube seen from the LD (Left-Down) viewpoint.}%
}%
\end{figure}

%%---------------------------------
\begin{quote}
\begin{verbatim}
\usepackage{tikz,rubikcube,rubikrotation}
...
\begin{figure}[hbt]
  \centering
  %% filename: rubikexampfig3.tex%
\RubikFaceUp{R}{O}{G}{R}{R}{O}{W}{W}{G}%
\RubikFaceDown{W}{B}{B}{R}{O}{B}{B}{R}{B}%
\RubikFaceLeft{G}{W}{O}{Y}{B}{O}{Y}{B}{B}%
\RubikFaceRight{Y}{Y}{Y}{Y}{G}{R}{R}{O}{W}%
\RubikFaceFront{G}{G}{O}{G}{W}{B}{R}{W}{Y}%
\RubikFaceBack{R}{W}{W}{G}{Y}{G}{O}{Y}{O}%

  \ShowCube{4cm}{0.8}{\DrawRubikCubeLD}
  \parbox{0.7\textwidth}{%
      \caption{....}%
  }%
\end{figure}
\end{verbatim}
\end{quote}

\noindent\textsc{note}: The command \verb!\SaveRubikState{}! is part of the 
\textsc{rubikrotation} package; see the documentation for further details.


\pagebreak

%----------------------
\subsection{Series of cubes}
\typeout{---Example (series of cubes)}

Here we show a convenient way of displaying a series of small cubes showing 
the action of a sequence of rotations (\rr{U}, \rr{R}, \rr{F}) on a solved cube. 

\begin{figure}[hbt]
\centering
\RubikCubeSolvedWY%
\ShowCube{2cm}{0.5}{\DrawRubikCubeRU}%
\quad\Rubik{U}\quad%
\RubikRotation{U}%
\ShowCube{2cm}{0.5}{\DrawRubikCubeRU}%
\quad\Rubik{R}\quad%
\RubikRotation{R}%
\ShowCube{2cm}{0.5}{\DrawRubikCubeRU}%
\quad\Rubik{F}\quad%
\RubikRotation{F}%
\ShowCube{2cm}{0.5}{\DrawRubikCubeRU}%
\caption{The rotations \rr{U}, \rr{R}, \rr{F}\ on a solved cube.} 
\end{figure}

{\noindent}The code for the above sequence is as follows:
\begin{quote}
\begin{verbatim}
\usepackage{tikz,rubikcube,rubikrotation}
...
\begin{figure}[hbt]
  \centering%
  \RubikCubeSolvedWY%
  \ShowCube{2cm}{0.5}{\DrawRubikCubeRU}%
  \quad\Rubik{U}\quad%
  \RubikRotation{U}%
  \ShowCube{2cm}{0.5}{\DrawRubikCubeRU}% 
  \quad\Rubik{R}\quad%
  \RubikRotation{R}%
  \ShowCube{2cm}{0.5}{\DrawRubikCubeRU}%
  \quad\Rubik{F}\quad%
  \RubikRotation{F}%
  \ShowCube{2cm}{0.5}{\DrawRubikCubeRU}%
\caption{The rotations \rr{U}, \rr{R}, \rr{F}\ on a solved cube.} 
\end{figure}
\end{verbatim}
\end{quote}

{\noindent}\textsc{Note} 1:  We start with the  White-opposite-Yellow (WY) 
solved cube, using the command  \verb!\RubikCubeSolvedWY!.
A White-opposite-Blue (WB) solved cube is available as \verb!\RubikCubeSolvedWB!.

\medskip

{\noindent}\textsc{Note} 2: There are four different commands for typesetting 
 3x3x3 rotation codes in various formats (a slightly different set of four 
commands are used  for the equivalent 2x2x2 rotation codes). 
In~this  (3x3x3) example, we have used two of them; for example: 
{\newline}\verb!\rr{U}! $\rightarrow$ \rr{U} \ (here the \verb!\rr{}! stands 
for `rubik rotation'),  and also 
{\newline}\verb!\Rubik{U}! $\rightarrow$ \Rubik{U}. This command typesets 
both the hieroglyph and the rotation code vertically one above the other.
 The other two commands are as follows:
{\newline}\verb!\rrh{U}! $\rightarrow$ \rrh{U} \ (here the \verb!\rrh{}! 
stands for `rubik rotation hieroglyph'),  and also 
{\newline}\verb!\textRubik{U}! $\rightarrow$ \textRubik{U}. This command 
typesets both the rotation code and the  hieroglyph on the same line one 
after the other.


\pagebreak

%----------------------
\subsection{Rotation sequence}
\typeout{---Example (rotation sequence)}

We now explore using the named Rubik cube rotation sequences and associated 
patterns available in the  \textsc{rubikpatterns} package---a small macro database 
(see the \textsc{rubikpatterns} documentation, and also its companion file \texttt{rubikpatternsLIST.pdf}). Having  the sequences available as  macros is 
very convenient since (a)~it avoids making errors when typing them out, 
and (b)~allows the easy application of software tools.
 
 A Rubik pattern  is the configuration generated  by a sequence of rotations 
 (or `moves') from some initial starting configuration  (typically a `solved' 
 configuration).  For example, FourSpot is a well known pattern which we can  
 generate  from a solved Rubik cube  using the macro \verb!\FourSpot!;  it is 
 defined in the \textsc{rubikpatterns} package as follows:
\begin{quote}
\begin{verbatim}
\newcommand{\FourSpot}{[FourSpot],F2,B2,U,Dp,R2,L2,U,Dp,<(12q*, 8f*)>}
\newcommand{\fourspot}{\FourSpot}
\end{verbatim}
\end{quote}
 Note that for convenience the macros names in the \textsc{rubikpatterns} package 
 are defined in  both upper and lower-case (i.e.,~the macros \verb!\FourSpot! 
 and \verb!\fourspot! are identical).  The following figure shows the FourSpot 
 sequence and pattern. 

\bigskip

\noindent%
\RubikCubeSolvedWY%
\ShowCube{2cm}{0.5}{\DrawRubikCubeLU}
\RubikRotation{\CornerRotation}
\quad%
\SequenceBraceA{CornerRotation}{%
                     \ShowSequence{}{\Rubik}{\SequenceLong}%
                     }
\quad%
\ShowCube{5cm}{0.5}{\DrawRubikCubeSF}

\bigskip

{\noindent}The code for the above figure is as follows:


\begin{quote}
\begin{verbatim}
\usepackage{tikz,rubikcube,rubikrotation,rubikpatterns}
...
\noindent%
\RubikCubeSolvedWY%
\ShowCube{2cm}{0.5}{\DrawRubikCubeRD}
\RubikRotation{\FourSpot}
\quad%
\SequenceBraceA{FourSpot}{%
                         \ShowSequence{}{\Rubik}{\SequenceLong}%
                        }
\quad%
\ShowCube{5cm}{0.5}{\DrawRubikCubeSF}
\end{verbatim} 
\end{quote}
Note that we have spread  the code slightly here in order to emphasise that 
the \verb!\ShowSequence! command is being used as an argument for the 
\verb!\SequenceBraceA! command (the `A' in the command \verb!\SequenceBraceA! 
denotes that the annotation is placed Above the sequence.)
We have used a solved cube with the WY (White opposite Yellow) 
configuration (\verb!\RubikCubeSolvedWY!).
The first cube is drawn from the  RD (Right-Down) viewpoint (\verb!\DrawRubikCubeRD!).
The second cube is drawn from the SF (Semi-Flat)  viewpoint (\verb!\DrawRubikCubeSF!) 
so we can see all the faces.

Note that we execute the \verb!\RubikRotation{}! command \textit{before} showing 
the sequence itself. This is because the Perl program (CALLed by 
\verb!\RubikRotation!) allocates the rotation sequence to the variable 
\verb!\SequenceLong! which we then display using the \verb!\ShowSequence! command 
(see the \textsc{rubikrotation} documentation for details).


\pagebreak 

%----------------------
\subsection{SixTs}
\typeout{---Example (SixTs)}

A  more interesting cube pattern  is the SixTs configuration (from the 
\textsc{rubikpatterns} package),  which we  now show in a slightly 
different way (adding an extra face and also some cube sidebars), as follows:

\bigskip

\noindent\hfil%
\RubikCubeSolvedWB
\ShowCube{1.6cm}{0.4}{
     \DrawRubikCubeRU  
     \DrawRubikFlatDown{0}{-3.3}}
\RubikRotation{\sixts}
\quad\SequenceBraceA{SixTs}{\ShowSequence{,\ }{\texttt}{\SequenceLong}}
\quad\ShowCube{3cm}{0.5}{
           \DrawRubikCubeRU
           \DrawRubikCubeSidebarFL{RU}
           \DrawRubikCubeSidebarFD{RU}
           \DrawRubikCubeSidebarBR{RU}
           \DrawRubikCubeSidebarBU{RU}
           }

\hfil

\bigskip

This time we have started with a solved cube having the WB  configuration 
(White opposite Blue), which is generated  using the command  
\verb!\RubikCubeSolvedWB! 
(we have added the \textsc{down} face (blue) below to reveal the colour of this 
face---see note below). 

The rotation  sequence is   in `long-format' (expanded into 
separate rotations), comma-separated and space,  typewriter font,  
using the command \verb!\ShowSequence{,\ }{\texttt}{\SequenceLong}!.

The final image shows the sixT cube  together with the main cube sidebars 
 indicating the colours of the  adjacent facelets, 
using the \verb!\DrawRubikCubeSidebarXX! commands. 
The code for the above figure is as follows:

\begin{quote}
\begin{verbatim}
\RubikCubeSolvedWB
\ShowCube{1.6cm}{0.4}{%
           \DrawRubikCubeRU%
           \DrawRubikFlatDown{0}{-3.3}%
           }
\RubikRotation{\SixTs}
\quad\SequenceBraceA{SixTs}{%
           \ShowSequence{,\ }{\texttt}{\SequenceLong}%
           }
\quad\ShowCube{2cm}{0.5}{
           \DrawRubikCubeRU
           \DrawRubikCubeSidebarFL{RU}
           \DrawRubikCubeSidebarFD{RU}
           \DrawRubikCubeSidebarBR{RU}
           \DrawRubikCubeSidebarBU{RU}
           }
\end{verbatim} 
\end{quote}



\subsubsection*{Notes}

\begin{enumerate} 

 \item   We have drawn the  \textsc{down} face of the first cube using the command
\verb!\DrawRubikFlatDown{0}{-3.3}! where the two arguments are  the $x$ and 
$y$ coordinates of the \textit{bottom left} corner of the \textsc{down} face  
(blue); these $(x,y)$ arguments allow us to position the blue `flat' face in 
relation to the cube. Note that the grid origin of  a cube image (on the page) 
coincides with the \textit{bottom left} corner of the \textsc{front} face 
(green in this case).

\item   The first image is really just 4 units wide. This is because the 2D width 
of the  \textit{side}  face (red) is designed to measure  1~unit wide in the 
oblique view (similarly, the 2D height of the  \textit{top} face also measures 
just 1~unit). Consequently, since the TikZ scale factor used is $0.4$, then 
the (minimum) width argument for its \verb!\ShowCube{}{}{}! command is 
$4\times 0.4 = 1.6$cm., hence we have \verb!\ShowCube{1.6cm}{0.4}{...}!.

\end{enumerate}


\pagebreak

%----------------------
\subsection[Cycle three edges]{Cycle three edges (flip two)}
\typeout{---Example (cycle three edges (flip two))}

The following example illustrates a sequence used for cycling three  edges, 
which is often used  in generating the `cross' formation while solving the 
final layer. However, it is a useful exercise to perform this sequence on 
a solved cube, as this  reveals quite clearly  its rather complex actions 
(it~also swaps two pairs of corners).

 The black arrow (no flip) and magenta arrows (flip) indicate  the \textsc{up} 
 face  edge cubies which are cycled anti-clockwise by the  sequence 
\ShowSequence{,}{\rr}{F,R,U,Rp,Up,Fp}. The blue arrows indicate the associated 
`collateral damage' actions (two pairs of corner cubies swap positions); 
depending on the circumstances, these other actions may be an  acceptable 
cost since they can be fixed at a later stage. (This diagram  is from 
Section~13 in the \textsc{rubikcube} package documentation).

Note that we  include the \textsc{up}-face sidebars to reveal the remaining 
facelet colours of the top layer (using the command \verb!\DrawRubikFaceUpSide!) 
so we can see how the  sequence makes the cubies rotate and turn as they move.

 \bigskip%
 \noindent\hfil%
 \RubikCubeSolvedWY%
 \ShowCube{2cm}{0.5}{\DrawRubikCubeRU}%
 \quad\ShowCube{2.3cm}{0.5}{%
    \DrawRubikFaceUpSide%
    \draw[thick,->,color=magenta] (1.5,0.5) -- (2.4, 1.4);
    \draw[thick,->] (2.5,1.5) -- (1.6, 2.4);
    \draw[thick,->,color=magenta] (1.3, 2.3) -- (1.3, 0.5);
    \draw[thick,<->,  color=blue] (0.5,2.6) -- (2.5, 2.6);
    \draw[thick,<->,  color=blue] (0.5,0.3) -- (2.5, 0.3);
    }%
 \RubikRotation{\CycleThreeEdgesFlipTwo}%
 \quad\ShowSequence{}{\Rubik}{\SequenceLong}\quad$\longrightarrow$\quad%
 \ShowCube{2.3cm}{0.5}{\DrawRubikFaceUpSide}%
 \hfil%
 \bigskip%


{\noindent}The code for the figure is as follows 


\begin{verbatim}
\newcommand{\CycleThreeEdgesFlipTwo}{F,R,U,Rp,Up,Fp}%
 ...
 ...
 \RubikCubeSolved%
 \ShowCube{2cm}{0.4}{\DrawRubikCubeRU}%
 \quad\ShowCube{2.3cm}{0.4}{%
    \DrawRubikFaceUpSide%
    \draw[thick,->,color=magenta] (1.5,0.5) -- (2.4, 1.4);
    \draw[thick,->] (2.5,1.5) -- (1.6, 2.4);
    \draw[thick,->,color=magenta] (1.3, 2.3) -- (1.3, 0.5);
    \draw[thick,<->,  color=blue] (0.5,2.6) -- (2.5, 2.6);
    \draw[thick,<->,  color=blue] (0.5,0.3) -- (2.5, 0.3);
    }%
 \RubikRotation{\CycleThreeEdgesFlipTwo}%
 \quad\ShowSequence{}{\Rubik}{\SequenceLong}\quad$\longrightarrow$\quad%
 \ShowCube{2.3cm}{0.4}{\DrawRubikFaceUpSide}%
\end{verbatim}

{\noindent}Note that the  TikZ  \verb!\draw! commands (for drawing the arrows 
on the cube face) start with a lower-case letter, and also require a terminal 
semicolon.




\pagebreak
%----------------------
\subsection{Superflip}
\typeout{---Example (superflip)}


Once you can solve Rubik's cube, then an interesting exercise  is to generate  
the so-called `superflip' configuration, in which all the 
corners are correctly solved, while all the edges are 
flipped\,\footnote{See the `superflip' entry in \textit{Wikipedia}, and also 
the Kociemba website (\url{www.kociemba.org/cube.htm}); particularly the page 
\url{http://kociemba.org/math/oh.htm}}. 

%--------draw the images-------------
\begin{figure}[hbt]
  \centering
  \RubikCubeSolvedWY%
  \ShowCube{2cm}{0.5}{\DrawRubikCubeRU}%
  \qquad$\longrightarrow$\qquad%
  \RubikRotation{\superflip}%
  \ShowCube{2cm}{0.5}{\DrawRubikCubeLD}%
  \qquad%
  \ShowCube{5cm}{0.5}{\DrawRubikCubeSF}%
\caption{Two representations of the superflip configuration.}
\end{figure}
%----------------------

{\noindent}A superflip sequence converts the solved cube on the left 
into the form on the right, using the command \verb!\RubikRotation{\superflip}!.
The  code for the above figure is shown below (the~\verb!\superflip! macro 
used here  is from the \textsc{rubikpatterns} package).
\begin{verbatim}
\usepackage{tikz,rubikcube,rubikrotation,rubikpatterns}
...
\begin{figure}[hbt]
  \centering
  \RubikCubeSolvedWY%
  \ShowCube{2cm}{0.5}{\DrawRubikCubeRU}%
  \qquad$\longrightarrow$\qquad%
  \RubikRotation{\superflip}%
  \ShowCube{2cm}{0.5}{\DrawRubikCubeLD}
  \qquad%
  \ShowCube{5cm}{0.5}{\DrawRubikCubeSF}
\caption{...}
\end{figure}
\end{verbatim}


The following superflip sequence\,\footnote{This particular  superflip 
sequence (in the \textsc{rubikpatterns} package) is due to Reid (1995); 
for details see the \textsc{rubikpatterns} package documentation, and 
also  \url{http://kociemba.org/math/oh.htm}. 
Another 20-move superflip sequence (due to H Kociemba),  is designated 
as  K32466  in  \url{www.nickalls.org/dick/papers/tex/RUBIK20moves.zip}.} 
has just 20  HTM rotations (Half Turn Metric: counting 180~degree turns 
as just one `rotation'). Note that the \textsc{rubikpatterns} package 
contains  this particular superflip sequence  as the macro \verb!\superflip!
(see \texttt{rubikpatterns.pdf}). Consequently  the  code 
\verb!\ShowSequence{,}{\large\texttt}{\superflip}!, will typeset the sequence
as follows:

\begin{center}
%% show the Short code
\ShowSequence{,}{\texttt}{\superflip}
\end{center}
Note that for convenience, the \textsc{rubikpatterns} package includes 
the sequence name (in square brackets)  as the first element of the 
associated  macro. This is possible since the contents of a comma-separated 
square bracket  are not actioned as a rotation by the \verb!\RubikRotation! 
command.   

\pagebreak

Next  we present the same superflip sequence but without commas  and in 
the form of hieroglyphs, using  the \verb!\Rubik! font, for which we 
require the `expanded' \verb!\SequenceLong!  form (since the `short form' 
includes trailing digits---see Section~11 in the \textsc{rubikcube} 
documentation), using the  following code:

\begin{quote}
\begin{verbatim}
\usepackage{tikz,rubikcube,rubikrotation,rubikpatterns}
...
\RubikCubeSolvedWY
\RubikRotation{\superflip}
\noindent\strut\hspace{-8mm}\ShowSequence{}{\Rubik}{\SequenceLong}
\end{verbatim}
\end{quote}
which gives


\bigskip

\RubikCubeSolvedWY
\RubikRotation{\superflip}
\noindent\strut\hspace{-8mm}\ShowSequence{}{\Rubik}{\SequenceLong}


%%---------------------------------
\bigskip

\subsubsection*{Equivalent sequence}


Interestingly,  the superflip sequence is actually equivalent to
\Rubikbrace{\Rubikbracket{\Rubik{Rm}\Rubik{U}}4, \Rubik{x}, 
\Rubik{yp}}3 \ \ (Harris (2008), p151).
Furthermore we can readily demonstrate this, as we can process 
this novel form of the sequence using  some useful features of 
the \verb!\RubikRotation! command, as follows:

\begin{quote}
\begin{verbatim}
\RubikCubeSolvedWF
\RubikRotation[3]{[superflip],(Rm,U)4,x,yp,<Harris (2008), p151>}
\ShowCube{4cm}{0.4}{\DrawRubikCubeSF}
\end{verbatim}
\end{quote}
which generates the following 

%-------------------
\begin{figure}[hbt]
  \centering
  \RubikCubeSolvedWY
   %%% \RubikRotation[3]{[shortsuper],(Rmp,Up)4,yp,x}% ORG
  \RubikRotation[3]{[shortsuper],(Rm,U)4,x,yp,<Harris (2008), p151>}
  \ShowCube{5cm}{0.5}{\DrawRubikCubeSF}
  \end{figure}
%----------------------
{\noindent}which is exactly the same configuration as before.
Note that to do this we made  use of  the `repeat'  option \verb![3]!  
as well as  the \verb!(Rmp,Up)4! `repeat-block' in 
the argument of  the \verb!\RubikRotation! command above. 


\pagebreak

%----------------------
\subsection{Inverse sequence}
\typeout{---Example (inverse sequence)}


Generating the inverse of a Rubik sequence involves (a)~reversing the 
order of the sequence, and (b)~inverting each rotation in the sequence 
(see Sections~5.1 and 5.1.1 in the \textsc{rubikrotation} package 
documentation).

From  the grey-cube example~(\ref{sec:greycube}) we saw that the sixspot 
sequence  is: \verb!U,Dp,R,Lp,F,Bp,U,Dp!; its inverse is therefore 
readily determined as \verb!D,Up,B,Fp,L,Rp,D,Up!. Note that this is 
easy to check since the sequence generated by the  \verb!\RubikRotation! 
command is held by the macro \verb!\SequenceLong!. For~example, the 
output of the following commands
\begin{quote}
\begin{verbatim}
\fbox{\strut\rule{0pt}{10pt}%
      The inverse of the sixspot sequence is:
      \RubikRotation{\sixspot,<inverse>}
      \ShowSequence{,}{\texttt}{\SequenceLong}.
    }
\end{verbatim}
\end{quote}
is \quad\fbox{\strut\rule{0pt}{10pt} %
The inverse of the sixspot sequence is: 
\RubikRotation{\sixspot,<inverse>}
\ShowSequence{,}{\texttt}{\SequenceLong}.
}

\bigskip

A sequence and its inverse will annihilate each other when applied 
consecutively.  For example, in the following figure we start with 
a solved cube and apply the sixspot sequence. Applying the  inverse 
of the sixspot sequence then results in the solved cube configuration
 again.

\newcommand{\sixspotarrow}{$\quad\overrightarrow{\strut\textsc{sixspot}}\quad$}
\newcommand{\sixspotINVarrow}{$\quad\overrightarrow{\strut\textsc{sixspotINV}}\quad$}

\begin{figure}[hbt]
\centering
\RubikCubeSolvedWY%  
   \ShowCube{2cm}{0.5}{\DrawRubikCubeRU}% 
\RubikRotation{\sixspot}%   
   \sixspotarrow%
   \ShowCube{2cm}{0.5}{\DrawRubikCubeRU}%
\RubikRotation{\sixspot,<inverse>}%  
   \sixspotINVarrow%
   \ShowCube{2cm}{0.5}{\DrawRubikCubeRU}
\end{figure}

{\noindent}The code for the above figure is as follows:

\begin{verbatim}
\newcommand{\sixspotarrow}{$\quad\overrightarrow{\strut\textsc{sixspot}}\quad$}
\newcommand{\sixspotINVarrow}{$\quad\overrightarrow{\strut\textsc{sixspotINV}}\quad$}
...
\RubikCubeSolvedWY\ShowCube{2cm}{0.5}{\DrawRubikCubeRU}% 
\RubikRotation{\sixspot}%   
   \sixspotarrow%
   \ShowCube{2cm}{0.5}{\DrawRubikCubeRU}%
\RubikRotation{\sixspot,<inverse>}%  
   \sixspotINVarrow%
   \ShowCube{2cm}{0.5}{\DrawRubikCubeRU}
\end{verbatim}


A significant property of the superflip configuration is that it is its own 
inverse. Consequently we can achieve  a similar result simply by applying the 
superflip sequence \textit{twice in succession}, as follows:

\newcommand{\superfliparrow}{$\quad\overrightarrow{\strut\textsc{superflip}}\quad$}

\begin{figure}[hbt]
\centering
\RubikCubeSolved%  
   \ShowCube{2cm}{0.5}{\DrawRubikCubeRU}\superfliparrow% 
\RubikRotation{\superflip}%   
   \ShowCube{2cm}{0.5}{\DrawRubikCubeRU}\superfliparrow%
\RubikRotation{\superflip}%  
   \ShowCube{2cm}{0.5}{\DrawRubikCubeRU}
\end{figure}


\pagebreak

%------------------- 
\subsection{OLL example}

% ThreeCube YellowCross (from dnRUBIKsolution15)
\newcommand{\yellowcross}{[yellowcross],F,R,U,Rp,Up,Fp}
\newcommand{\arrow}[1]{$\overrightarrow{\strut\mbox{#1}}$}
\newcommand{\ShowYellowFace}{\ShowCube{2.6cm}{0.6}{\NoSidebar{X}\DrawRubikFaceUpSide}}

Here we illustrate the use of the \verb!\NoSidebar! command (see the 
\textsc{rubikcube} package documentation) to generate nice displays 
of so-called OLL (Orientate Lower Layer) configurations, which are 
typically rendered using the yellow face.

For example, the following sequence shows an efficient 
algorithm (sequence) for `solving'  the OLL configuration shown 
in cube~A (listed as OLL case~33 in Harris 
(2008,  page~61)\footnote{Harris D (2008). \textit{Speedsolving the cube}, 
(Stirling Publishing Co, New York) {\newline}(\texttt{www.stirlingpublishing.com}) 
[covers 2x2x2, 3x3x3, 4x4x4, 5x5x5  cubes].}  and also on the Harris  OLL webpage  \url{http://www.cubestation/3x3x3/cfop/oll/ollprintablepage.php}).


%% Dan Harris case 33 (in his 2008 book)
\newcommand{\CaseXXXIII}{[caseHarris33],R, U, Rp, U, R, Up, Rp, Up, Rp,  F, R, Fp}
%  {\DH2008thth}

\bigskip

%% setup OLL position DH33 (Dan Harris)
\RubikCubeGreyAll
\RubikFaceUp{X}{Y}{Y}
            {Y}{Y}{X}
            {Y}{X}{X}
\RubikFaceBack  XXY XXXXXX
\RubikFaceLeft  XXX XXXXXX 
\RubikFaceRight YYX XXXXXX 
\RubikFaceFront XYX XXXXXX
%
\cubenumber{A}%
\ShowCube{2.6cm}{0.6}{\NoSidebar{X}\DrawRubikFaceUpSide}%
%
\RubikRotation{\CaseXXXIII}%
\quad\SequenceBraceA{OLLcase33Harris}{%
                \ShowSequence{}{\Rubik}{\CaseXXXIII}%
                }%
\quad$\longrightarrow$\quad%
\cubenumber{B}%
\ShowCube{2.6cm}{0.6}{\NoSidebar{X}\DrawRubikFaceUpSide}%

\bigskip

{\noindent}The code for the above figure is as follows (note that 
the \verb!\NoSidebar{X}! command is placed inside the \verb!\ShowCube! 
environment in order to limit its action).

\begin{quote}
\begin{verbatim}
\newcommand{\cubenumber}[1]{\strut\raisebox{1cm}{#1}}
\newcommand{\CaseXXXIII}{[caseHarris33],R,U,Rp,U,R,Up,Rp,Up,Rp,F,R,Fp}
%
\RubikCubeGreyAll
\RubikFaceUp{X}{Y}{Y}
            {Y}{Y}{X}
            {Y}{X}{X}
\RubikFaceBack  XXY XXXXXX
\RubikFaceLeft  XXX XXXXXX 
\RubikFaceRight YYX XXXXXX 
\RubikFaceFront XYX XXXXXX
\cubenumber{A}%
\ShowCube{2.6cm}{0.6}{\NoSidebar{X}\DrawRubikFaceUpSide}%
\RubikRotation{\CaseXXXIII}%
\quad\SequenceBraceA{OLLcase33Harris}{%
                \ShowSequence{}{\Rubik}{\CaseXXXIII}%
                }%
\quad$\longrightarrow$\quad%
\cubenumber{B}%
\ShowCube{2.6cm}{0.6}{\NoSidebar{X}\DrawRubikFaceUpSide}%
\end{verbatim}
\end{quote}
Note that since we originally defined the OLL case~33 algorithm 
(Harris 2008)  as a macro, it follows that we could instead have 
recovered the  OLL~33 configuration (cube~A above) using the 
inverse algorithm, as follows: 

\bigskip

\RubikCubeGreyAll
\RubikFaceUpAll{Y}
\RubikRotation{<inverse>,\CaseXXXIII}
\ShowCube{2.6cm}{0.6}{\NoSidebar{X}\DrawRubikFaceUS}%
\hspace{1cm}
\begin{minipage}{4cm}
\begin{verbatim}
\RubikCubeGreyAll
\RubikFaceUpAll{Y}
\RubikRotation{<inverse>,\CaseXXXIII}
\ShowCube{2.6cm}{0.6}{\NoSidebar{X}\DrawRubikFaceUS}
\end{verbatim}
\end{minipage}

\bigskip

{\noindent}Note that in this last example we have used the more convenient  
short-hand US (UpSide) version of the \verb!\DrawRubikFaceUpSide! command, 
namely \verb!\DrawRubikFaceUS!.



\pagebreak
%----------------------
\subsection{Cube patterns}
\typeout{---Example (Cube patterns)}

\newcommand{\CycleThreeUpEdges}{[CycleUpEdges],F2,U,Rm,Up,Up,Lm,U,Fp2}
\newcommand{\StripesOrtho}{[StripesOrtho],Lp,Fp,Rp,L,U,D, Rp,Dp,Lp,Bp,U2,Lp,Up,R,L,Up,F2,Rp,<Harris 2008, p150>}

We now look at how the \verb!\RubikRotation! command can be used to 
investigate  cube  patterns, which are typically shown only as a 
regular cube. For example, consider the following interesting sequence  
\linebreak\texttt{L' F' R' L U D R' D' L' B' U2 L' U' R L U' F2 R'} 
(Harris 2008, page~150), which is presented as a cube with orthogonal 
stripes (see figure~A). In order to show  the pattern using the Rubik
 bundle, it is often useful to first express the sequence as a macro. 
 In this particular case we need to  adjust the Harris (2008) notation 
 slightly to make it compatible (ie, insert commas, change trailing 
 `primes' into a trailing `p'), give it a suitable name 
 (say, \verb![StripesOrtho]!), and, for completion, include the citation 
in an `infoblock' ($<$...$>$), as follows:

\begin{verbatim}
\newcommand{\StripesOrtho}{%
   [StripesOrtho],Lp,Fp,Rp,L,U,D,Rp,Dp,Lp,Bp,U2,Lp,Up,R,L,Up,F2,Rp,<Harris 2008, p150>} 
\end{verbatim}
We can now  process it simply by using its macro-name  as the argument for 
the  \verb!\RubikRotation! command; for example, the following code displays 
it as figure~A. 

\begin{verbatim}
\RubikCubeSolved
\RubikRotation{\StripesOrtho}
\ShowCube{2.6cm}{0.6}{\DrawRubikCubeRU}
\end{verbatim}

At first sight this looks like a nice example of orthogonal stripes, 
but if we look at all the faces  by rendering the cube  in a Semi-Flat 
(SF) view (using the command \verb!\DrawRubikCubeSF!)
 ---see~figure~B--- this reveals that unfortunately only five faces 
actually have a center stripe.

\bigskip

\RubikCubeSolved
\RubikRotation{\StripesOrtho}
\cubenumber{A}
\ShowCube{2.6cm}{0.6}{\DrawRubikCubeRU}
\hspace{3cm}
\cubenumber{B}
\ShowCube{4cm}{0.4}{\DrawRubikCubeSF}

\bigskip

{\noindent}So the question is: can we improve on this and create a cube 
with a fully orthogonal pattern, by  swapping the relevant red/yellow 
and blue/yellow edge cubies in the back layer? Since it is impossible 
to swap just two edge cubies within the same layer in a 3x3x3 cube 
without changing any other cubies, the answer is no. However, we are 
able to cycle cleanly three edge cubies within one face, and a single  
cycle could at least create the final stripe with only minimal collateral
damage.

For example, if we rotate the whole cube using the rotation denoted 
as  \rr{Rcp} (= \verb!\rr{Rcp}!)\,\footnote{There are several equivalent
notations for this: \rr{Rcp} $\equiv$ \rr{xp} $\equiv$ \rr{CRp}; 
the c and C denote a whole cube rotation.} to bring the \textsc{back} face to 
the \textsc{up} position (as shown in figure~C), we can then cycle the three 
\textsc{up} edge cubies UL, UR, UF clockwise using the following
`CycleThreeUpEdges' sequence:
\begin{verbatim}
\newcommand{\CycleThreeUpEdges}{F2,U,Rm,Up,Up,Lm,U,Fp2} 
\end{verbatim}
We can combine these two actions using the command 
\verb!\RubikRotation{Rcp,\CycleThreeUpEdges}!, which will then align the 
three red facelets in the \textsc{up} face  and form the final stripe 
(see figure~D). 

\RubikCubeSolved
\RubikRotation{\StripesOrtho,Rcp} 
\cubenumber{C}
\ShowCube{2.6cm}{0.6}\DrawRubikCubeRU
\hspace{3cm}
\RubikRotation{\CycleThreeUpEdges}
\cubenumber{D}
\ShowCube{4cm}{0.4}\DrawRubikCubeSF

\medskip

The downside, of course, is that a yellow and a white facelet are also 
swapped in the top layer, but at least we now have a cube with six 
mutually adjacent orthogonal central stripes\,!


\pagebreak

%%=====================TWOcube stuff======================================
%%========================================================================

\section{TwoCube examples (2x2x2)} 

\subsection{Commands \& notation} 

Regarding the 2x2x2 cube as a `TwoCube' (instead of  a RubikCube) makes 
the 2x2x2 commands very intuitive (assuming you are familiar with the
3x3x3 commands), since they are mostly obtained just by exchanging 
the word `Rubik' in  3x3x3 commands for the word `Two'. For example,

\medskip
  
\verb!3x3x3!\quad\hspace{1cm}$\rightarrow$\quad\hspace{1cm}\verb!2x2x2!  
 
\smallskip

\verb!\RubikCubeSolvedWY!   $\rightarrow$  \verb!\TwoCubeSolvedWY!

\verb!\RubikRotation!\qquad\ $\rightarrow$ \quad\verb!\TwoRotation!

\verb!\SaveRubikState!\qquad\ $\rightarrow$  \quad\verb!\SaveTwoState! 

\medskip

Note that  although the 2x2x2 cube does not have any middle slices, one can still
 actually use the 3x3x3 commands \verb!\RubikRotation! and \verb!\SaveRubikState! 
 in the usual way (because the corner cubies are still processed in 
the usual way). However, in the interests of consistency 
both of these  commands have  equivalent `Two' versions: namely
\verb!\TwoRotation! and  \verb!\SaveTwoState!.

Of course there are considerably fewer rotation  codes and hieroglyphs 
associated with the TwoCube (since there are no middle slices), and 
their commands also mirror the 3x3x3 equivalents in  the \textsc{rubikcube} 
package; i.e.,~\verb!Rubik! $\rightarrow$ \verb!Two! and 
\verb!r! $\rightarrow$ \verb!t!.  Thus the changes  associated with the 
four key commands  can be summarised as  follows : 

 \medskip
  
 \verb!3x3x3! \hphantom{$\rightarrow$} \verb!2x2x2!  
 
 \smallskip
 
 \verb!\rr{}! $\rightarrow$ \verb!\tr{}!

 \verb!\rrh{}! $\rightarrow$ \verb!\trh{}!

 \verb!\Rubik{}! $\rightarrow$ \verb!\Two{}!

 \verb!\textRubik{}! $\rightarrow$ \verb!\textTwo{}!


%-----------------------

\bigskip

%------------------- 
\subsection{Visualising white-space}

This example shows  how the  F version of the \verb!\ShowCube{}{}{}! command can be useful for revealing excess white space.


 
 \bigskip 
 
\TwoCubeSolvedWB% 
\ShowCubeF{3cm}{0.6}{\DrawTwoCubeRU}%
\quad\ShowCubeF{1.7cm}{0.6}{\DrawTwoFaceUS}%
\hspace{1cm}
\begin{minipage}{0.3\textwidth}
\begin{verbatim}
\TwoCubeSolvedWB% 
\ShowCubeF{3cm}{0.6}{\DrawTwoCubeRU}%
\quad%
\ShowCubeF{1.7cm}{0.5}{\DrawTwoFaceUS}%
\end{verbatim}
\end{minipage}
 
 
\bigskip 
 
 Both images above have  an F-box drawn around them (so we can see the
 extent of any white space) using the  F version of the  \verb!\ShowCube!$\{\mbox{\textit{width}}\}\{\mbox{\textit{scale-factor}}\}\{\}$ command, 
 namely \verb!\ShowCubeF{}{}{}!.  Clearly the \verb!3cm! width we have 
 used for the first cube above is too much, and so in the second pair 
 of images below  we have reduced this width for the first image  to 
 its correct value of \verb!1.6cm!, and have removed the F from the 
 \verb!\ShowCubeF! command accordingly.

\bigskip 
 
\quad% 
\TwoCubeSolvedWB% 
\ShowCube{1.6cm}{0.6}{\DrawTwoCubeRU}%
\quad%
\ShowCube{1.7cm}{0.6}{\DrawTwoFaceUS}%
\hspace{1cm}
\begin{minipage}{0.3\textwidth}
\begin{verbatim}
\TwoCubeSolvedWB% 
\ShowCube{1.6cm}{0.6}{\DrawTwoCubeRU}%
\quad%
\ShowCube{1.7cm}{0.5}{\DrawTwoFaceUS}%
\end{verbatim}
\end{minipage}

\bigskip

Thus use of the F version of the the \verb!ShowCube! command allows 
the user to easily determine---by trial \& error---the correct minipage 
width  for a particular cube. 


\pagebreak 

%------------------- 
\subsection{Flat and semi-flat formats}

We can view the hidden faces of a cube using the semi-flat (SF) and 
flat (F) versions of the \verb!\DrawTwoCube..! command as follows.

In the first figure we have also used the F version of the  
\verb!\ShowCube{}{}{}! command to reveal the size of the minipage 
which contains the image. Notice that since the 2D width of the side 
face (red) is approximately 1 unit, we can estimate that the total 
width of the image  will be 6 + 1 = 7 units, and hence as we have 
specified a TikZ scale factor of 0.6, then the  required width of 
the minipage is approximately $4.2$ ($= 7 \times 0.6$) 
---see Section~4.1 in the \textsc{rubikcube} package documentation 
for details).

\bigskip 
 
\begin{minipage}{0.3\textwidth}
\centering
\TwoCubeSolvedWB% 
\ShowCubeF{4.2cm}{0.6}{\DrawTwoCubeSF}%
\end{minipage}
    \hspace{2cm}
\begin{minipage}{0.3\textwidth}
\begin{verbatim}
\TwoCubeSolvedWB% 
\ShowCubeF{4.2cm}{0.6}{\DrawTwoCubeSF}%
\end{verbatim}
 \end{minipage}

\bigskip 
 
\begin{minipage}{0.3\textwidth}
\centering
\TwoCubeSolvedWB% 
\ShowCube{4.8cm}{0.6}{\DrawTwoCubeF}%
\end{minipage}
    \hspace{2cm}
\begin{minipage}{0.3\textwidth}
\begin{verbatim}
\TwoCubeSolvedWB% 
\ShowCube{4.8cm}{0.6}{\DrawTwoCubeF}%
\end{verbatim}
 \end{minipage}

 
 
 \pagebreak
 
%------------------- 
\subsection{Grey cube}


A cube with all of the facelets coloured grey (a `grey' cube) can be 
useful as a starting state for instruction. Such a cube is generated 
by the command \verb!\TwoCubeGreyAll!.
We can then  allocate colours to those facelets demonstrating 
some feature of  cube manipulation. 
 Here we are showing how to rotate the GWR corner cubie in the bottom layer
 up into its `solved' position  in the top layer. We also use two cube 
 sidebars  to show the colours of the hidden facelets of the corner cubie.
  
  \bigskip
  

\noindent\hfil%
\TwoCubeGrey
\ShowCube{2cm}{0.7}{\DrawTwoCube}%
%
\TwoFaceUp{X}{X}
          {R}{W}%
\TwoFaceFront{G}{R}
             {W}{X}%
\TwoFaceLeft XX
             XG%
\TwoFaceDown{R}{X}
            {X}{X}%         
\quad\ShowCube{2cm}{0.7}{
        \DrawTwoCube
        \DrawTwoCubeSidebarFD{RU}
        \DrawTwoCubeSidebarFL{RU}
        }%
\quad\Two{Lp}\quad
\TwoRotation{Lp}
\ShowCube{2cm}{0.7}{\DrawTwoCube}%
\hfil%

\bigskip

{\noindent}The code for the above figure is as follows:
\begin{quote}
\begin{verbatim}
\noindent\hfil%
\TwoCubeGrey
\ShowCube{2cm}{0.6}{\DrawTwoCube}%
%
\TwoFaceUp{X}{X}
          {R}{W}%
\TwoFaceFront{G}{R}
             {W}{X}%
\TwoFaceLeft XX
             XG%
\TwoFaceDown{R}{X}
            {X}{X}%         
\ShowCube{2cm}{0.6}{
        \DrawTwoCube
        \DrawTwoCubeSidebarFD{RU}
        \DrawTwoCubeSidebarFL{RU}
        }%
\quad\Two{Lp}\quad
\TwoRotation{Lp}
\ShowCube{2cm}{0.6}{\DrawTwoCube}%
\hfil%
\end{verbatim}
\end{quote}
Notice that the curly brackets associated with the \verb!\TwoFace..! commands 
can be omitted providing at least one space is left before the first colour code 
(see the \verb!\TwoFaceLeft..! command above).

 
 \pagebreak
 
%------------------- 
\subsection{Scrambled cube}

In the following example we scramble the cube using 20 random rotations using  
$\langle$\texttt{random,20}$\rangle$ as the argument 
for the \verb!\TwoRotation{}! command. We have also added a set of cube sidebars
to the second cube. The final semi-flat (SF) cube has the same colour configuration 
and shows the hidden faces, allowing you to check the colours of the sidebars.
 
\bigskip
          
 \noindent\hfil%
 \TwoCubeSolvedWY%
 \ShowCube{1.8cm}{0.7}{\DrawTwoCubeRU}%
 \qquad$\longrightarrow$\qquad%
 %---------------
 \TwoRotation{random,20}
 %----------------
 \ShowCube{2.8cm}{0.7}{
    \DrawTwoCubeRU
    \DrawTwoCubeSidebarFL{RU} % Front-Left edge sidebar
    \DrawTwoCubeSidebarFD{RU} % Front-Down edge
    \DrawTwoCubeSidebarBR{RU} % Back-Right edge
    \DrawTwoCubeSidebarBU{RU} % Back-Up edge
    }%
 \qquad\ShowCube{5cm}{0.7}{\DrawTwoCubeSF}%
 \hfil%

 \bigskip
 
 
 \begin{quote}
 \begin{verbatim}
 \noindent\hfil%
 \TwoCubeSolvedWY%
 \ShowCube{1.8cm}{0.7}{\DrawTwoCubeRU}%
 \qquad$\longrightarrow$\qquad%
 %---------------
 \TwoRotation{random,20}
 %----------------
 \ShowCube{2.8cm}{0.7}{
    \DrawTwoCubeRU
    \DrawTwoCubeSidebarFL{RU} % Front-Left edge sidebar
    \DrawTwoCubeSidebarFD{RU} % Front-Down edge
    \DrawTwoCubeSidebarBR{RU} % Back-Right edge
    \DrawTwoCubeSidebarBU{RU} % Back-Up edge
    }%
 \qquad\ShowCube{5cm}{0.7}{\DrawTwoCubeSF}%
 \hfil%
  \end{verbatim}
  \end{quote}

 
 
%------------------- 
\subsubsection*{A simple scramble}

A simple scramble (from a solved position)  which 
makes each face have four different facelets is $\left\{\mbox{\tr{R}, 
\tr{y}}\right\}$5 as follows (note that the  optional \verb!\TwoRotation! 
square bracket  holds the repeat number~$5$): 

\bigskip

\begin{minipage}{0.3\textwidth}
\centering
\TwoCubeSolvedWB% 
\TwoRotation[5]{R,y}%
\ShowCube{5cm}{0.6}{\DrawTwoCubeF}%
\end{minipage}
    \hspace{2cm}
\begin{minipage}{0.3\textwidth}
\begin{verbatim}
\TwoCubeSolvedWB% 
\TwoRotation[5]{R,y}%
\ShowCube{5cm}{0.6}{\DrawTwoCubeF}%
\end{verbatim}
 \end{minipage}


\pagebreak
 
%------------------- 
\subsection{Swap two corners}

The following example is the header figure for the \textsc{rubiktwocube} 
package documentation, and shows the  `SwapTwoCorners' sequence  often  
necessary in the solution of a 2x2x2 cube.
 
 \bigskip

 \noindent\hfil%
 %% set up the cube with corners swapped
 \TwoCubeSolvedWY% 
 \TwoRotation{Rc2}%  turn upside down (= Rc2) x2
 \TwoRotation{\swaptwocorners}%
 %%
 % now show the cube, before and after swapping the corners again
 \ShowCube{1.6cm}{0.6}{\DrawTwoCubeRU}%
 \quad\ShowCube{1.7cm}{0.5}{\DrawTwoFaceUS}%
 \TwoRotation{\swaptwocorners}%
 \quad\SequenceBraceA{SwapTwoCorners}{\ShowSequence{}{\Two}{\SequenceLong}}%
 \quad$\longrightarrow$%
 \quad\ShowCube{1.7cm}{0.5}{\DrawTwoFaceUS}%
  \hfil%
 
 \bigskip%

 {\noindent}The code for the above figure  is as follows:
 \begin{verbatim}
 \usepackage{tikz}
 \usepackage{rubikcube,rubikrotation,rubikpatterns,rubiktwocube}
 ...
 \newcommand{\SwapTwoCorners}{[swaptwocorners],Rp,F,Rp,B2,R,Fp,Rp,B2,R2,Up}
 \newcommand{\swaptwocorners}{\SwapTwoCorners}
 ... 
 \noindent
 \hfil%
 %% set up the cube with corners swapped
 \TwoCubeSolvedWY% 
 \TwoRotation{Rc2}%  turn cube upside down about the R face
 \TwoRotation{\swaptwocorners}%
 %% now show the cube, before and after swapping the corners
 \ShowCube{1.6cm}{0.6}{\DrawTwoCubeRU}%
 \quad\ShowCube{1.7cm}{0.5}{\DrawTwoFaceUS}%
 \TwoRotation{\swaptwocorners}%
 \quad\SequenceBraceA{SwapTwoCorners}{\ShowSequence{}{\Two}{\SequenceLong}}%
 \quad$\longrightarrow$%
 \quad\ShowCube{1.7cm}{0.5}{\DrawTwoFaceUS}%
 \hfil%
 \end{verbatim}

\pagebreak
 
%------------------- 
\subsection{OLL example}

Here we illustrate an OLL algorithm which orientates the last layer
starting from a single yellow facelet in the UFL position. 

\bigskip

\noindent\hfil%
\TwoCubeGreyAll%
\TwoFaceUp{X}{X}%
          {Y}{X}%
\TwoFaceBack  XY XX
\TwoFaceRight XY XX%
\TwoFaceFront XY XX%
\ShowCube{2cm}{0.7}{\NoSidebar{X}\DrawTwoFaceUS}
    \SaveTwoState{cubeTWO.tex} %***
\TwoRotation{\allyellow}%
\quad\SequenceBraceA{AllYellow}{\ShowSequence{}{\Two}{\SequenceLong}}\quad$\longrightarrow$\quad%
\ShowCube{1.5cm}{0.7}{\NoSidebar{X}\DrawTwoFaceUS}
\hfil%

\bigskip

The code is shown below. Note that we have used the \verb!\NoSidebar{X}! 
command to disable the drawing of any grey sidebars  by the \verb!\DrawTwoFaceUS! 
command\footnote{Note that the \textbackslash\texttt{DrawTwoFaceUS} command is a short-hand 
equivalent to the \textbackslash\texttt{DrawTwoFaceUpSide} command.} 
 and hence only yellow sidebars are drawn. Note that we have placed the 
\verb!\NoSidebar{X}! command inside the \verb!\ShowCube! environment in order 
to limit its action locally.

\begin{quote}
\begin{verbatim}
\noindent\hfil%
\TwoCubeGreyAll%
\TwoFaceUp{X}{X}%
          {Y}{X}%
\TwoFaceBack  XY XX
\TwoFaceRight XY XX%
\TwoFaceFront XY XX%
\ShowCube{2cm}{0.7}{\NoSidebar{X}\DrawTwoFaceUS}%
\TwoRotation{\allyellow}%
\quad\SequenceBraceA{AllYellow}{\ShowSequence{}{\Two}{\SequenceLong}}%
\quad$\longrightarrow$\quad%
\ShowCube{1.5cm}{0.7}{\NoSidebar{X}\DrawTwoFaceUS}%
\hfil% 
\end{verbatim}
\end{quote}

\bigskip


\begin{center}
------ END ------
\end{center}

\end{document}